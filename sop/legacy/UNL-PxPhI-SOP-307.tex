\documentclass[12pt]{unlsilabsop}
\title{Cold box}
\date{August 6, 2015}
\author{Frank Meier Aeschbacher}
\approved{Frank Meier Aeschbacher}
\sopid{307}
\sopversion{v1}
\sopabstract{This document describes the regular maintenance for the cold box.}
\begin{document}

\maketitle

%------------------------------------------------------------------
\section{Scope}
This document describes the maintenance of the cold box and all its parts.

%------------------------------------------------------------------
\section{Purpose}
The cold box is a crucial equipment for testing modules. This SOP describes the maintenance steps for keeping it in working condition.

%------------------------------------------------------------------
%\section{Definitions}

%------------------------------------------------------------------
\section{Responsibilities}
Every person using the probe station is responsible to maintain it. While a minimum schedule for maintenance activities is outlined in this document, any person is allowed to trigger out-of-cycle maintenance if needed to maintain working conditions.

%------------------------------------------------------------------
\section{Equipment}

\begin{itemize}
    \item Distilled or deionized water
\end{itemize}

% Consider adding images of key equipment if it helps

%------------------------------------------------------------------
\section{Procedure}

% Mention to record certain observations if this is needed
\subsection{Weekly maintenance}
\begin{enumerate}
    \item Inspect the water level of the chiller. It should be within the marks. If the water level is too low, fill it up with distilled or deionized water.
\end{enumerate}

\subsection{Monthly maintenance}
\begin{enumerate}
    \item Inspect the interior of the cold box:
    \begin{itemize}
        \item Open the cold box cover.
        \item Inspect the tubes for any leaks (traces of water, condensation).
        \item Inspect the cables to the relays. No discolorization or change in color should be visible.
    \end{itemize}
    \item Inspect the HV connection between the power supply (``Keithley'') and the DTB for any obvious damage. Replace damaged cables.
    \item Inspect exterior of cold box for any obvious damage, e.g.~cabling, tubing etc.
\end{enumerate}

%------------------------------------------------------------------
\section{Documentation}
The following information needs to be recorded in the report for the UNL logbook:
\begin{itemize}
    \item Date, time (start--end) and operator name.
    \item Any special observations, e.g.~any deviation from expectation.
    \item Any action taken.
\end{itemize}

\end{document}

