%%%%%%%%%%%%%%%%%%%%%%%%%%%%%%%%%%%%%%%%%%%%%%%%%%%%%%%%%%%%%%%%%%%%%
% LaTeX Template: Project Modified (v 0.2) by jluo
%
% Original Source: http://www.howtotex.com
% Date: February 2019
% 
% This is a title page template which be used for articles & reports.
% 
% 
%%%%%%%%%%%%%%%%%%%%%%%%%%%%%%%%%%%%%%%%%%%%%%%%%%%%%%%%%%%%%%%%%%%%%%

\documentclass[11pt]{article}
\usepackage[a4paper]{geometry}
\usepackage[myheadings]{fullpage}
% \usepackage{fancyhdr}
\usepackage{lastpage}
\usepackage{graphicx, wrapfig, subcaption, setspace, booktabs}
\usepackage[T1]{fontenc}
\usepackage[font=small, labelfont=bf]{caption}
\usepackage[protrusion=true, expansion=true]{microtype}
\usepackage{sectsty}
\usepackage{url, lipsum}
% \usepackage{tgbonum}
\usepackage{hyperref}
\usepackage{xcolor}
% \usepackage{lscape}
\usepackage{pdflscape}
\usepackage{color}
 
\hypersetup{
    colorlinks=true,       % false: boxed links; true: colored links
    urlcolor=blue           % color of external links
}


\newcommand{\HRule}[1]{\rule{\linewidth}{#1}}
\onehalfspacing
\setcounter{tocdepth}{5}
\setcounter{secnumdepth}{5}



%-------------------------------------------------------------------------------
% HEADER & FOOTER
%-------------------------------------------------------------------------------
%\pagestyle{fancy}
%\fancyhf{}
%\setlength\headheight{15pt}
%\fancyhead[L]{Student ID: 1034511}
%\fancyhead[R]{Anglia Ruskin University}
%\fancyfoot[R]{Page \thepage\ of \pageref{LastPage}}
%-------------------------------------------------------------------------------
% TITLE PAGE
%-------------------------------------------------------------------------------

\begin{document}
{\fontfamily{cmr}\selectfont
\title{ \normalsize \textsc{}
		\\ [2.0cm]
		\HRule{0.5pt} \\
		\LARGE \textbf{\uppercase{Extra Guidelines for UNL HEP Lab during COVID-19}
		\HRule{0.5pt} \\ [0.5cm]
		\normalsize \today \vspace*{5\baselineskip}}
		}

\date{}

\author{
		Caleb Fangmeier \\
		Frank Golf \\ 
		Ilya Kravchenkco \\ 
		University of Nebraska-Lincoln\\}

\maketitle
%\newpage
%\tableofcontents
%\newpage

%-------------------------------------------------------------------------------
% Section title formatting
\sectionfont{\scshape}
%-------------------------------------------------------------------------------

%-------------------------------------------------------------------------------
% BODY
%-------------------------------------------------------------------------------

\newpage
\section{Guidelines during COVID-19}
Until further notice, all personnel should work from home if possible.  Work in the lab should be performed only if it cannot be done elsewhere.  In the event that you must work in the HEP lab (JH 174), including the clean room and general area, the following guidance is in effect.  This is in addition to safety and good scientific practices outlined in the general laboratory documentation.

\begin{itemize}
\item If you feel you are even slightly sick, whether your symptoms match COVID-19 or not, do not come to the lab.
\item Do not come to the lab unless you must. If work can be performed from home, you should do so.  If you are not sure, talk to your supervisor.
\item No more than 3 people are allowed in the lab at the same time.
\item Undergraduate and graduate students are not allowed to work in the lab alone.
\item If you decide that you need to come to the lab, you must do the following in advance:
    \begin{itemize}
    \item Add to the lab google calendar the day(s) and time window(s) you plan to be in the lab.  The title should be your name.  This must be done at least the day before you plan to go in.
    \item In addition, send email to Frank Golf and Caleb Fangmeier, contact info at the bottom, and notify them at least the day before of your plans to work in the lab and the following information:
    
    	    \begin{itemize}
	    	\item When you want to go in and for how long
		\item The reason you need to go in
	    \end{itemize}

	\item If your supervisor or one of the lab technicians tells you to not go to the lab, you must not go to the lab.
    \end{itemize}
\item To assist in maintaining a safe work place, the following items are available in the laboratory:
	\begin{itemize}
	\item gloves: available in several sizes
	\item masks: \textcolor{red}{check requirements}
	\item soap: \textcolor{red}{check, get some if needed}
	\item towels: \textcolor{red}{check, get some if needed}
	\item sanitizer: \textcolor{red}{check availability with university}
	\item wipes: \textcolor{red}{check availability with university}
	\end{itemize}
\newpage
\item While working in the lab, the following practices must be followed for everyone's safety:
	\begin{itemize}
		\item Wash your hands with soap immediately upon entering and just before exiting the laboratory.  Do this every time you enter or leave the laboratory.
		\item Practice social distancing whenever possible.  Ideally, you should work at least 10 feet apart from another person.
		\item A mask must be properly worn at all times when working in the lab if not alone.
		\item Wear disposable gloves when permitted by the work you're doing.  Remember, gloves must be worn in the clean room at all times.
		\item After you are done working in a space or when you leave the laboratory for the day, use a disinfectant spray or wipes to 	clean all surfaces that you touched including the bench top, tools, keyboard, your custom-made equipment, your chair, etc.
		\item Wipe the door handles before you leave the lab.
		\item Please dispose of whatever refuse you produce.
	\end{itemize}
\end{itemize}

If you have questions, please ask. 

Contact information for the laboratory contacts is below.

\begin{itemize}
\item Caleb Fangmeier
	\begin{itemize}
	\item email: \href{mailto:cfangmeier74@gmail.com}{cfangmeier74@gmail.com}
	\item phone: 
	\end{itemize}
\item Frank Golf
	\begin{itemize}
	\item email: \href{mailto:fgolf@unl.edu}{fgolf@unl.edu}
	\item phone: (858) 395 6869
	\end{itemize}
\item Ilya Kravchenko
	\begin{itemize}
	\item email: \href{mailto:ikrav@unl.edu}{ikrav@unl.edu}
	\item phone: 
	\end{itemize}	
\end{itemize}


%-------------------------------------------------------------------------------
% Appendices
%-------------------------------------------------------------------------------

\appendix

%-------------------------------------------------------------------------------
% REFERENCES
%-------------------------------------------------------------------------------
% \newpage
% \section*{References}

%[2]John W. Eaton, David Bateman, Sren Hauberg, Rik Wehbring (2015). GNU
%Octave version 4.0.0 manual: a high-level interactive language for numer-
%ical computations. Available: http://www.gnu.org/software/octave/doc/
%interpreter/. 
}
\end{document}

%-------------------------------------------------------------------------------
% SNIPPETS
%-------------------------------------------------------------------------------

%\begin{figure}[!ht]
%	\centering
%	\includegraphics[width=0.8\textwidth]{file_name}
%	\caption{}
%	\centering
%	\label{label:file_name}
%\end{figure}

%\begin{figure}[!ht]
%	\centering
%	\includegraphics[width=0.8\textwidth]{graph}
%	\caption{Blood pressure ranges and associated level of hypertension (American Heart Association, 2013).}
%	\centering
%	\label{label:graph}
%\end{figure}

%\begin{wrapfigure}{r}{0.30\textwidth}
%	\vspace{-40pt}
%	\begin{center}
%		\includegraphics[width=0.29\textwidth]{file_name}
%	\end{center}
%	\vspace{-20pt}
%	\caption{}
%	\label{label:file_name}
%\end{wrapfigure}

%\begin{wrapfigure}{r}{0.45\textwidth}
%	\begin{center}
%		\includegraphics[width=0.29\textwidth]{manometer}
%	\end{center}
%	\caption{Aneroid sphygmomanometer with stethoscope (Medicalexpo, 2012).}
%	\label{label:manometer}
%\end{wrapfigure}

%\begin{table}[!ht]\footnotesize
%	\centering
%	\begin{tabular}{cccccc}
%	\toprule
%	\multicolumn{2}{c} {Pearson's correlation test} & \multicolumn{4}{c} {Independent t-test} \\
%	\midrule	
%	\multicolumn{2}{c} {Gender} & \multicolumn{2}{c} {Activity level} & \multicolumn{2}{c} {Gender} \\
%	\midrule
%	Males & Females & 1st level & 6th level & Males & Females \\
%	\midrule
%	\multicolumn{2}{c} {BMI vs. SP} & \multicolumn{2}{c} {Systolic pressure} & \multicolumn{2}{c} {Systolic Pressure} \\
%	\multicolumn{2}{c} {BMI vs. DP} & \multicolumn{2}{c} {Diastolic pressure} & \multicolumn{2}{c} {Diastolic pressure} \\
%	\multicolumn{2}{c} {BMI vs. MAP} & \multicolumn{2}{c} {MAP} & \multicolumn{2}{c} {MAP} \\
%	\multicolumn{2}{c} {W:H ratio vs. SP} & \multicolumn{2}{c} {BMI} & \multicolumn{2}{c} {BMI} \\
%	\multicolumn{2}{c} {W:H ratio vs. DP} & \multicolumn{2}{c} {W:H ratio} & \multicolumn{2}{c} {W:H ratio} \\
%	\multicolumn{2}{c} {W:H ratio vs. MAP} & \multicolumn{2}{c} {\% Body fat} & \multicolumn{2}{c} {\% Body fat} \\
%	\multicolumn{2}{c} {} & \multicolumn{2}{c} {Height} & \multicolumn{2}{c} {Height} \\
%	\multicolumn{2}{c} {} & \multicolumn{2}{c} {Weight} & \multicolumn{2}{c} {Weight} \\
%	\multicolumn{2}{c} {} & \multicolumn{2}{c} {Heart rate} & \multicolumn{2}{c} {Heart rate} \\
%	\bottomrule
%	\end{tabular}
%	\caption{Parameters that were analysed and related statistical test performed for current study. BMI - body mass index; SP - systolic pressure; DP - diastolic pressure; MAP - mean arterial pressure; W:H ratio - waist to hip ratio.}
%	\label{label:tests}
%\end{table}%\documentclass{article}
%\usepackage[utf8]{inputenc}

%\title{Weekly Report template}
%\author{gandhalijuvekar }
%\date{January 2019}

%\begin{document}

%\maketitle

%\section{Introduction}

%\end{document}
