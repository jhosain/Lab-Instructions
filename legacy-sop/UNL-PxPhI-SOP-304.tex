\documentclass[12pt]{unlsilabsop}
\title{Probe station}
\date{August 6, 2015}
\author{Frank Meier Aeschbacher}
\approved{Frank Meier Aeschbacher}
\sopid{304}
\sopversion{v1}
\sopabstract{This document describes the regular maintenance for the probe station.}
\begin{document}

\maketitle

%------------------------------------------------------------------
\section{Scope}
This document describes the maintenance of the probe station.

%------------------------------------------------------------------
\section{Purpose}
The probe station is a crucial equipment for testing and inspecting modules. This SOP describes the maintenance steps for keeping it in working condition.

%------------------------------------------------------------------
%\section{Definitions}

%------------------------------------------------------------------
\section{Responsibilities}
Every person ousing the probe station is responsible to maintain it. While a minimum schedule for maintenance activities is outlined in this document, any person is allowed to trigger out-of-cycle maintenance if needed to maintain working conditions.

%------------------------------------------------------------------
\section{Equipment}

\begin{itemize}
    \item Wipe pads, non-dusty tissues
    \item 2-Propanol
\end{itemize}

% Consider adding images of key equipment if it helps

%------------------------------------------------------------------
\section{Procedure}

% Mention to record certain observations if this is needed
\subsection{Weekly maintenance}
\begin{enumerate}
    \item Clean the surface of the probe station (especially the sample table) mounted to it using 2-Propanol and wiping pads.
    \item Remove any obvious dirt. 
    \item Inspect the tips of the two picoprobes. The should be straight and no obvious damage is visible.
    \item Check the cleanliness of the oculars of the microscope. Clean it if dirty.
\end{enumerate}

%------------------------------------------------------------------
\section{Documentation}
The following information needs to be recorded in the report for the UNL logbook:
\begin{itemize}
    \item Date, time (start--end) and operator name.
    \item Any special observations, e.g.~any deviation from expectation.
    \item Any action taken.
\end{itemize}

\end{document}

