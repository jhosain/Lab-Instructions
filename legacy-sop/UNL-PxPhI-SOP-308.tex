\documentclass[12pt]{unlsilabsop}
\title{HV power supply}
\date{August 6, 2015}
\author{Frank Meier Aeschbacher}
\approved{Frank Meier Aeschbacher}
\sopid{308}
\sopversion{v1}
\sopabstract{This document describes the regular maintenance for the HV power supply.}
\begin{document}

\maketitle

%------------------------------------------------------------------
\section{Scope}
This document describes the maintenance of the HV power supply (``Keithley'').

%------------------------------------------------------------------
\section{Purpose}
The HV power supply is a crucial equipment for testing and inspecting modules. This SOP describes the maintenance steps for keeping it in working condition.

%------------------------------------------------------------------
%\section{Definitions}

%------------------------------------------------------------------
\section{Responsibilities}
Every person using a HV power supply is responsible to maintain it. While a minimum schedule for maintenance activities is outlined in this document, any person is allowed to trigger out-of-cycle maintenance if needed to maintain working conditions.

%------------------------------------------------------------------
\section{Equipment}

\begin{itemize}
    \item Voltmeter
\end{itemize}

% Consider adding images of key equipment if it helps

%------------------------------------------------------------------
\section{Procedure}

% Mention to record certain observations if this is needed
\subsection{Weekly maintenance}
\begin{enumerate}
    \item Inspect the power supply on the exterior for any obvious damage.
    \item Set the output to -100\,V and the compliance limit to 1\,mA. Connect a voltmeter and check the voltage. If the reading is within the precision of the voltmeter, the power supply is good\footnote{If working properly, the HV power supplies are far superior in measuring the voltage than any regular voltmeter. This test is only to get a general idea if high voltage is present.}
\end{enumerate}

%------------------------------------------------------------------
\section{Documentation}
The following information needs to be recorded in the report for the UNL logbook:
\begin{itemize}
    \item Date, time (start--end) and operator name.
    \item Any special observations, e.g.~any deviation from expectation.
    \item Any action taken.
\end{itemize}

\end{document}

