\documentclass[12pt]{unlsilabsop}
\title{Gantry}
\date{August 6, 2015}
\author{Frank Meier Aeschbacher}
\approved{Frank Meier Aeschbacher}
\sopid{302}
\sopversion{v1}
\sopabstract{This document describes the regular maintenance for the gantry.}
\begin{document}

\maketitle

%------------------------------------------------------------------
\section{Scope}
This document describes the maintenance of the gantry.

%------------------------------------------------------------------
\section{Purpose}
The gantry is a crucial equipment for manufacturing modules. This SOP describes the maintenance steps for keeping it in working condition.

%------------------------------------------------------------------
%\section{Definitions}

%------------------------------------------------------------------
\section{Responsibilities}
Every person operating the gantry is responsible to maintain it. While a minimum schedule for maintenance activities is outlined in this document, any person is allowed to trigger out-of-cycle maintenance if needed to maintain working conditions.

%------------------------------------------------------------------
\section{Equipment}

\begin{itemize}
    \item Wipe pads, non-dusty tissues
    \item 2-Propanol
    \item Grease, safe for cleanroom applications.

    There is a small tube in the cabinet, supplied by the manufacturer of the gantry. Only use grease recommended by the manufacturer.
\end{itemize}

% Consider adding images of key equipment if it helps

%------------------------------------------------------------------
\section{Procedure}

% Mention to record certain observations if this is needed
\subsection{Weekly maintenance}
\begin{enumerate}
    \item Clean the surface of the gantry and the tools mounted to it using 2-Propanol and wiping pads.
    \item Remove any obvious dirt. In case of residues from encapsulation, make sure the encapsulant is sufficiently cured before removing it by hand.
\end{enumerate}

\subsection{Monthly maintenance}
\begin{enumerate}
    \item Check the steel surface of the left rail. If it looks dry, apply a small amount of grease using a wiping pad.

    Note: Other rails do not need to be checked as they are not exposed like the left rail.
\end{enumerate}
No other maintenance is required. If the gantry is malfunctioning for no obvious reason, get in touch with the manufacturer (Aerotech).

%------------------------------------------------------------------
\section{Documentation}
The following information needs to be recorded in the report for the UNL logbook:
\begin{itemize}
    \item Date, time (start--end) and operator name.
    \item Any special observations, e.g.~any deviation from expectation.
    \item Any action taken.
\end{itemize}

\end{document}

